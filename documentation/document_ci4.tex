\documentclass[a4paper,12pt]{article}
\usepackage{fullpage}
\usepackage{amsmath,amsthm,amsfonts,amssymb,amscd}
\usepackage{xcolor}
\usepackage{graphicx}
\usepackage{adjustbox}
\usepackage{geometry}
\usepackage{caption}
\usepackage{xepersian}
\usepackage{multicol}
\usepackage{listings}
\usepackage{color}
\usepackage{hyperref}
\usepackage{bidi} 
\usepackage{enumitem}

% Colors
\definecolor{titlepagecolor}{cmyk}{0.75,0.68,0.67,0.90} % Cover background
\definecolor{CustomAccent}{HTML}{2BAB8C} % Accent color for English text
%\definecolor{CustomBackground}{HTML}{1C1C1C} % Background for content pages
\definecolor{CustomBackground}{cmyk}{0.75,0.68,0.67,0.90}% Background for content pages

%%%%%%%%%%%%%%%%%%%%%%%%%%%%%%%%%%%%%%%%%%%%%%%%%%%%%%%%%%

\definecolor{codebg}{cmyk}{0.75,0.68,0.67,0.90} % same as CustomBackground
\definecolor{accent}{HTML}{2BAB8C} % same as CustomAccent
\definecolor{codegray}{rgb}{0.8,0.8,0.8}
\definecolor{codegreen}{rgb}{0.4,1,0.4}
\definecolor{codepurple}{rgb}{1,0.6,1}
\definecolor{keywordcolor}{rgb}{1,0.3,0.6}

\lstdefinestyle{darkstyle}{
	backgroundcolor=\color{codebg},   
	commentstyle=\color{codegreen},
	keywordstyle=\color{keywordcolor},
	numberstyle=\tiny\color{codegray},
	stringstyle=\color{codepurple},
	basicstyle=\ttfamily\footnotesize\color{white},
	breakatwhitespace=false,         
	breaklines=true,                 
	captionpos=b,                    
	keepspaces=true,                 
	numbers=left,                    
	numbersep=10pt,                  
	showspaces=false,                
	showstringspaces=false,
	showtabs=false,                  
	tabsize=4,
	frame=single,
	rulecolor=\color{accent}
}

\lstset{style=darkstyle}

%%%%%%%%%%%%%%%%%%%%%%%%%%%%%%%%%%%%%%%%%%%%%%%%%%%%%%%%%%%%







% Persian and Latin fonts
\settextfont{Vazir.ttf}[BoldFont = Vazir-Bold.ttf, Path = fonts/]
\setlatintextfont{Times New Roman}

% Line spacing
\renewcommand{\baselinestretch}{1.2}
\renewcommand{\thesection}{\arabic{section})}

\color{white}


% Homework number
\newcommand{\HomeworkNumber}{1}

% Cover-only settings
\pagenumbering{gobble}

% ---------- COVER PAGE ----------
\begin{document}
	\begin{latin}
		\begin{titlepage}
			\newgeometry{top=1in,bottom=1in,right=0in,left=0in}
			\thispagestyle{empty}
			\pagecolor{titlepagecolor}
			\color{white}
			\begin{center}
				\vspace*{\stretch{1}}
				
				{\fontsize{48}{0}\bfseries\selectfont \color{CustomAccent} COMPUTATIONAL INTELLIGENCE}
				
				\vskip 1.5\baselineskip
				{\fontsize{24}{0}\selectfont PROJECT 4 DOCUMENTATION}
				
				\vspace*{\stretch{2}}
				\adjincludegraphics[width=1\paperwidth]{assets/cover2.png}
				
				\vspace*{\stretch{2}}
				{\fontsize{20}{0}\selectfont \color{CustomAccent}
					Ferdowsi University of Mashhad \\
					Department of Computer Engineering
				}
				
				\vskip 1.5\baselineskip
				{\Large SPRING 2025}
				
				\vspace*{\stretch{1}}
			\end{center}
		\end{titlepage}
	\end{latin}
	
	% ---------- RESET PAGE SETTINGS ----------
	\clearpage
	\nopagecolor
	\pagecolor{CustomBackground}
	\color{white}
	\newgeometry{top=1in,bottom=1in,left=1in,right=1in}
	\pagenumbering{arabic}
	
	% ---------- HEADER (PERSIAN) ----------
	\hrule \medskip
	\begin{minipage}{0.295\textwidth}
		\raggedleft \color{CustomAccent}
		مبانی هوش محاسباتی\\
		دانشگاه فردوسی مشهد\\
		گروه مهندسی کامپیوتر
	\end{minipage}
	\begin{minipage}{0.4\textwidth}
		\centering 
		\includegraphics[scale=0.3]{assets/fum-logo.png}
	\end{minipage}
	\begin{minipage}{0.295\textwidth} \color{CustomAccent}
		داکیومنت پروژه 4 \\
		دکتر فضل ارثی \\
		بهار 1404
	\end{minipage}
	\medskip\hrule
	\bigskip	
	
	%%%%%%%%%%%%%%%%%%%%%%%%%%%%%%%%%%%%%%%%%%%%%%%%%%%%%%%%%%%%%%%%%%%%%%
	
	\begin{table}[h]
		\centering
		\begin{tabular}{|c|c|}
			\hline
			\textbf{نام و نام خانوادگی} & \textbf{شماره دانشجویی} \\
			\hline
			امیرحسین افشار & 4012262196 \\
			\hline
						علیرضا صفار & 4011262281 \\
			\hline
		\end{tabular}
	\end{table}
	

%	\begin{figure}[h]
	%		\centering
	%		\includegraphics[scale=0.35]{assets/template.png}
	%		\caption*{\textcolor{CustomAccent}{k-means}}
	%	\end{figure}

	
\section{فاز اول}
\section*{فاز اول: استخراج ویژگی ها از مدل resnet18}

در ابتدا یک بررسی بر روی مدل resnet18 که با استفاده از pytorch پیاده سازی شده، انجام می دهیم:


	
\begin{table}[h]
	\centering
	\begin{latin}
	\begin{tabular}{|l|c|c|c|}
		\hline
		\textbf{Layer} & \textbf{\#Channels} & \textbf{Width} & \textbf{Height} \\
		\hline
		conv1 & 64 & 112 & 112 \\
		\hline
		bn1 & 64 & 112 & 112 \\
		\hline
		relu & 64 & 112 & 112 \\
		\hline
		maxpool & 64 & 56 & 56 \\
		\hline
		layer1 & 64 & 56 & 56 \\
		\hline
		layer2 & 128 & 28 & 28 \\
		\hline
		layer3 & 256 & 14 & 14 \\
		\hline
		layer4 & 512 & 7 & 7 \\
		\hline
		avgpool & 512 & 1 & 1 \\
		\hline
		fc & \multicolumn{3}{c|}{1000} \\
		\hline
	\end{tabular}
	\end{latin}	
	\caption{بلاک های مدل resnet18}
	\label{tab:resnet18}
\end{table}
%
%بدین ترتیب، میتوانیم تعداد فیچر های هر کدام از مراحل خواسته شده را پیدا کنیم:
%فیچر های ابتدایی: (تا لایه maxpool قبل از لایه اول):
%
%64 * 64 * 112 = 802,816
%
%فیجرهای میانی: (تا بلاک دوم):
%
%128 * 28 * 28 = 100,352
%
%فیجرهای سطح بالا: (تا قبل از fc): 
%
%512 * 1* 1 = 512
بدین ترتیب، می ‌توانیم تعداد فیچرهای هر کدام از مراحل خواسته شده را پیدا کنیم: 
\begin{table}[h]
	\centering
	\begin{tabular}{|c|c|c|c|}
		\hline
		\textbf{نوع فیلتر} & \textbf{تنظیمات} & \textbf{تعداد فیچرها} & \textbf{جزئیات بیشتر} \\
		\hline
		فیلترهای ابتدایی 
		&  $112 \times 64 \times 64$ & 802,816 & 
		تا لایه maxpool قبل از لایه اول 
		\\
		\hline
		فیلترهای میانی
		 & $28  \times 28 \times 128$ & 100,352 & 
		 تا بلاک دوم \\
		\hline
		فیلترهای سطح بالا &  $1 \times 1 \times 512$ & 512 & تا قبل از fc \\
		\hline
	\end{tabular}
	\caption{ویژگی های ابتدایی، ویژگیهای میانی، ویژگی های سطح بالا}
\end{table}
	

\pagebreak
برای هر کدام از این سه دسته فیچرها، برای این که درک بهتری از نحوه پراکندگی و corrolation آنها داشته باشیم، با استفاده از PCA و t-SNE یک نمایش کلی بدست آورده ایم. بدین منظور، فیچرهای استخراج شده از مدل resnet را به شکل زیر پلات کرده ایم:

\begin{figure}[h]
	\centering
	\includegraphics[scale=0.38]{assets/plot1.png}
	\caption{\textcolor{CustomAccent}{ویژگی های ابتدایی}}
\end{figure}

\begin{figure}[h]
	\centering
	\includegraphics[scale=0.38]{assets/plot2.png}
	\caption{\textcolor{CustomAccent}{ویژگی های میانی}}
\end{figure}

\begin{figure}[h]
	\centering
	\includegraphics[scale=0.38]{assets/plot3.png}
	\caption{\textcolor{CustomAccent}{ویژگی های سطح بالا}}
\end{figure}

همانطور که مشخص است، هرچه که از ویژگی های ابتدایی به ویژگی های سطح بالا عبور می کنیم، نمایش و بازنمایی بهتری از فیچرها به دست می آید که مطابق با انتظار است. بنابراین هم برای مدل های ساده فاز اول و هم برای مدل های stacked فاز دوم، انتظار داریم که برای فیچرهای سطح بالا، به دقت بالاتری دست پیدا کنیم. 


\pagebreak
\section*{فاز اول: مدل های ساده}
برای پیاده سازی مدل های ساده، 5 مدل زیر را در نظر گرفتیم:
\begin{latin}
\begin{itemize}
	\item SVM
	\item Logistic Regression
	\item Random Forest
	\item KNN
	\item Decision Tree
\end{itemize}
\end{latin}

\begin{table}[h]
	\centering
	\begin{latin}
	\begin{tabular}{|l|c|c|c|c|}
		\hline
		 \textbf{Classifier} & \textbf{Accuracy} & \textbf{Precision} & \textbf{Recall} & \textbf{F1-Score} \\
		\hline
		 SVM & 0.943 & 0.944 & 0.943 & 0.943 \\
		\hline
		 Logistic Regression & 0.967 & 0.967 & 0.967 & 0.967 \\
		\hline
		 Random Forest & 0.951 & 0.951 & 0.951 & 0.951 \\
		\hline
		 KNN & 0.943 & 0.945 & 0.943 & 0.942 \\
		\hline
		 Decision Tree & 0.820 & 0.825 & 0.820 & 0.821 \\
		\hline
	\end{tabular}
	\end{latin}
	\caption{دقت طبقه بند ها به ازای فیچر های high}
	\label{tab:classifier_performance}
\end{table}


\begin{table}[h]
	\centering
	\begin{latin}
	\begin{tabular}{|l|c|c|c|c|}
		\hline
		\textbf{Classifier} & \textbf{Accuracy} & \textbf{Precision} & \textbf{Recall} & \textbf{F1-Score} \\
		\hline
		SVM & 0.770 & 0.778 & 0.770 & 0.773 \\
		\hline
		Logistic Regression & 0.803 & 0.806 & 0.803 & 0.800 \\
		\hline
		Random Forest & 0.607 & 0.615 & 0.607 & 0.610 \\
		\hline
		KNN & 0.451 & 0.478 & 0.451 & 0.454 \\
		\hline
		Decision Tree & 0.533 & 0.530 & 0.533 & 0.531 \\
		\hline
	\end{tabular}
	\end{latin}
	\caption{دقت طبقه بند ها به ازای فیچر های سطح متوسط}
	\label{tab:classifier_performance_2}
\end{table}

\begin{table}[h]
	\centering
	\begin{latin}
	\begin{tabular}{|l|c|c|c|c|}
		\hline
		\textbf{Classifier} & \textbf{Accuracy} & \textbf{Precision} & \textbf{Recall} & \textbf{F1-Score} \\
		\hline
		SVM & 0.607 & 0.597 & 0.607 & 0.597 \\
		\hline
		Logistic Regression & 0.656 & 0.652 & 0.656 & 0.636 \\
		\hline
		Random Forest & 0.672 & 0.673 & 0.672 & 0.660 \\
		\hline
		KNN & 0.500 & 0.588 & 0.500 & 0.434 \\
		\hline
		Decision Tree & 0.492 & 0.485 & 0.492 & 0.484 \\
		\hline
	\end{tabular}
	\end{latin}
	\caption{دقت طبقه بند ها به ازای فیچر های سطح initial}
	\label{tab:classifier_performance_3}
\end{table}
%\begin{figure}[ht]
%	\centering
%	\begin{minipage}[t]{0.32\textwidth}
%		\centering
%		\includegraphics[width=\linewidth]{assets/template.png}
%		\caption{\textcolor{CustomAccent}{another caption}}
%	\end{minipage}
%	\hfill
%	\begin{minipage}[t]{0.32\textwidth}
%		\centering
%		\includegraphics[width=\linewidth]{assets/template.png}
%		\caption{\textcolor{CustomAccent}{caption new}}
%	\end{minipage}
%	\vspace{1em}
%	\hfill
%	\begin{minipage}[t]{0.32\textwidth}
%		\centering
%		\includegraphics[width=\linewidth]{assets/template.png}
%		\caption{\textcolor{CustomAccent}{caption new}}
%	\end{minipage}
%	\vspace{1em}
%\end{figure}

	
	
\clearpage
\section*{منابع}
\begin{LTR}
	\begin{latin}
		\begin{enumerate}[left=0pt,labelsep=5pt,itemsep=0pt,parsep=0pt,topsep=0pt]
			\item Image Denoising Algorithms: A Comparative Study of Different Filtration Approaches Used in Image Restoration.  \url{https://ieeexplore.ieee.org/abstract/document/6524379/}
			\item Digital Image Processing By Gonzalez 4th  \url{https://elibrary.pearson.de/book/99.150005/9781292223070}
			\item Automatic identification of noise in ice images using statistical features \url{https://www.researchgate.net/figure/Simple-pattern-classifier-to-identify-noise-types-of-Gaussian-Speckle-and -Salt-Pepper_tbl1_258714501}
			\item medium: A Beginners Guide to Computer Vision (Part 4)- Pyramid \url{https://medium.com/analytics-vidhya/a-beginners-guide-to-computer-vision-part-4-pyramid-3640edeffb00}
			
			
			\item medium: A Beginners Guide to Computer Vision (Part 4)- Pyramid \url{https://medium.com/analytics-vidhya/a-beginners-guide-to-computer-vision-part-4-pyramid-3640edeffb00}
			
			\item medium: A Beginners Guide to Computer Vision (Part 4)- Pyramid \url{https://medium.com/analytics-vidhya/a-beginners-guide-to-computer-vision-part-4-pyramid-3640edeffb00}
			
			\item medium: A Beginners Guide to Computer Vision (Part 4)- Pyramid \url{https://medium.com/analytics-vidhya/a-beginners-guide-to-computer-vision-part-4-pyramid-3640edeffb00}
			
		\end{enumerate}
	\end{latin}
\end{LTR}



\end{document}
